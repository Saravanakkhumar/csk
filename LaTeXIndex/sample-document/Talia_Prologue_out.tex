<L|a|t|e|x|C|o|m|m|e|n|t|C|n|t|-|0><L|a|t|e|x|C|o|m|m|e|n|t|C|n|t|-|2|9><L|a|t|e|x|C|o|m|m|e|n|t|C|n|t|-|6|7><L|a|t|e|x|C|o|m|m|e|n|t|C|n|t|-|9|3>

<L|a|t|e|x|C|o|m|m|e|n|t|C|n|t|-|1|2|2>

<L|a|t|e|x|C|o|m|m|e|n|t|C|n|t|-|1|4|7>

<L|a|t|e|x|C|o|m|m|e|n|t|C|n|t|-|1|6|6>

<a|d|d|c|h|a|p|C|n|t|-|1|8|1>

Nowadays, algorithms \pageinfoStart{1}deeply regulate the relationship of human beings with reality. These automatic procedures encode billions of operations recorded into the digital memories of computers, smartphones, and other ``intelligent'' digital objects. Whenever we need their services, these digital machines execute \hbox{algorithms} at enormous speeds. In their modern form as software programs, algorithms serve and guide us in the thousands of actions we perform daily with our digital devices. According to a general definition, an algorithm is a step-by-step method of solving a problem, of accomplishing a task. However, despite its very far-reaching diffusion, today the term ``algorithm'' has become a more specialized word. It indicates an unambiguous specification of an automatic procedure expressing a finite sequence of instructions including calculation, data processing, and abstract logical operations that solve a class of problems.

When humans \pageinfoStart{2}evolved, once the five senses developed, language became the mediator of human relationships with their own species and with the world in which they lived. Without language, prehistoric man was alone in facing the world. What could be known was only related to what happened to an individual. No other concept could be provided. Thanks to the development of orality, human beings were able to share and transfer their experiences and knowledge from one individual to another. Thus, they were able to learn not only from their own experiences but also from the experiences of their fellow man with whom they lived or came into contact with. The spoken language extended the reality of every human being who also learned about the world through what was described and related to them.

<m|a|r|k|b|o|t|h|C|n|t|-|1|9|4|2>

Subsequently, signs, symbols, and \pageinfoStart{3}other means of permanently recording experiences through thought and artistic forms changed the relationship between individuals and the world. All these human expressions are based on the drafting of visual forms or texts that respect a shared and communicable syntax. They allowed not only the accumulation of knowledge beyond what was possible through oral narrative and the memory of human beings, but also its sharing and diffusion in space and time. The invention of writing made it possible to build new relationships between human beings and reality, generated by what people were able to write and read.

From wax \pageinfoStart{4}tablets to papyrus to books, different forms of writing have regulated the lives of peoples and nations. Gutenberg's press and the various modern forms of communication have increasingly modified the experience of individuals and mediated their relationship with the world. Newspapers, TV, cinema, and different traditional and post-modern art forms have defined a new space of action and thought for human beings. They have also influenced and reshaped their ways of living.

Despite the \pageinfoStart{5}many and complex means of experience representation that human beings have invented, today the main mediation between people and the outside world is accomplished by algorithms. Their operational form is embodied in the software programs running on digital systems. The computers and digital devices that populate our world, such as smartphones, webcams, sensors, and smartwatches, are powered by algorithms. Their algorithms help people at work and at home when they need to plan a trip or solve challenging science and engineering problems. Operating inside digital hardware, they constantly surround us and interact with our daily lives.

The list \pageinfoStart{6}of examples of algorithms exploited for performing scientific tasks and solving complex scientific problems is almost endless. In the past, most of them were not feasible without computers. In the chapters of this book, several significant examples are discussed. To give a flavor of the new challenges that are faced with the help of algorithms, we cite the call to action launched by White House in March 2020. This call is based on the analysis of the text of more than 30,000 scholarly papers on coronavirus features for finding insights into treating COVID-19 and controlling the pandemic. Researchers from a few US research institutions released the Open Research Dataset CORD-19 of scholarly literature about COVID-19, severe acute respiratory syndrom coronavirus 2 (SARS-CoV-2), and the coronavirus group. The dataset represents the most extensive machine-readable coronavirus literature collection available for data and text mining to date. Text mining, natural language processing, and artificial intelligence algorithms applied on the COVID-19 literature provided novel solutions to accelerate the path to treatment and vaccines for COVID-19. However, it is impossible for people to manually go through 500,000 scientific papers and synthesize their findings. Advances in algorithm and Big Data technology are the most powerful tools in this scenario to help find answers to a key set of questions about a global threat such as COVID-19. This example, like many others, shows the power of algorithms in scientific discovery processes.

In the \pageinfoStart{7}last five centuries, science and engineering have been based on equations. Due to the discovery and definition of important equations, beginning in the time of Galileo and Newton, physics, mathematics, chemistry, and other scientific fields have made impressive progress. Scientists have solved many complex problems that advanced science and helped humans to improve their quality of life. We know that equations can be expressed as simple algorithms. However, algorithms may codify more complex models, calculations, and methods than equations. This is why the science of the new millennium will be based on algorithms that express advanced procedures and solutions for complex tasks that cannot be described by equations.

Algorithms play \pageinfoStart{8}a key role in science for new discoveries, in business for selling goods, and in finance to drive stock markets. However, they can also be designed and used to cause disruption for citizens, governments, and public administrations. Some well-known cases are the social bots spread by Russian agencies, the Cambridge Analytica scandal, and the threatening and risky social ranking \hbox{system} set up by the government in China. Social robots, or simply bots, are software applications that run automated tasks over social media. Social bots are based on a set of algorithms designed to mimic human behaviors for conversations, with behavioral patterns similar to that of a human user. They are often used for commercial goals. However, several cases have demonstrated that these algorithms have been used for political campaigns and for spreading unverified information throughout social media platforms. For example, the Atlantic Council's Digital Forensic Research Lab provided well-documented examples of Russian misinformation campaigns from Armenia to France and from Germany to the United States. The campaigns were conducted by using algorithms implementing social bots that sent thousands of posts or tweets simulating the behavior of a large number of supporters that never existed. In fact, algorithms that run social bots involve highly automated social accounts. Such accounts disseminate large quantities of political messages, and they are so accurately programmed that their targets (real individuals) never realize they're interacting with algorithms.

The 2015--2016 US \pageinfoStart{9}presidential election campaign was a crucial moment in the evolution of algorithmic methods for spreading political propaganda via social media. Initially, the owners of the social platforms failed to realize what they were up against. When Facebook was first asked how the Russian government influenced the election of Donald Trump, Mark Zuckerberg dismissed such foreign interference as negligible. Only a year after the election, in late October 2017, did Facebook reveal that Russia's messages and ads spread by their algorithms had in fact reached more than one hundred million Facebook\vadjust{\pagebreak} users. Moreover, a \hbox{secretive} Russian company linked to the Kremlin, named the Internet Research Agency, posted roughly 100,000 posts of troublesome content on Facebook. The posts reached around 30 million Americans before and during the election campaign.

A similar \pageinfoStart{10}scenario was shaped by the algorithms implemented by Cambridge Analytica, a British company that developed data collection and social media algorithms for influencing political preferences and public opinion. In 2018, \hbox{newspapers} reported that Cambridge Analytica had collected and used the personal data of many Facebook users. The personal data of up to 87 million users were acquired via the 270,000 users who used a Facebook quiz app called ``This Is Your Digital Life.'' By giving this third-party app permission to acquire their data, it also gave the algorithms of this app access to information on the user's friends network. This resulted in a database of about 87 million users being collected. The majority of these users had not explicitly given Cambridge Analytica permission to access their data. Indeed, the developers of the algorithm breached Facebook's terms of service by providing the data to Cambridge Analytica. This company designed algorithms implementing data analysis and profile building on top of the collected data. Then social bots supporting Trump's campaign generated a massive flow of posts in favor of this candidate and against his opponents.

This is \pageinfoStart{11}just another case where the role and influence of algorithms involved a large community of people, and could influence political and social organizations. The China Social Credit System that will be discussed extensively in the book provides another significant example of the massive coercive impact of algorithms. The People's Republic of China is setting up a large ranking system that, by using a collection of algorithms, will supervise the behavior of most of its very large population. This system ranks Chinese people based on their ``social credit.'' The program was fully operational nationwide by 2021, and it is already being used on millions of people across the country. The social credit program is based on a unified record system for individuals, businesses, and public administrations to be tracked and evaluated for credibility. Algorithms drive the system by raising or \hbox{lowering} a person's social score depending on their behavior. For example, algorithms of the Social Credit System can limit people's rights and put limits on services and luxury options they can access. Newspapers report that China has already started punishing people by restricting their travel. Nine million people with low scores have been blocked from buying tickets for domestic flights. Up to three million Chinese people are barred from getting business-class train tickets. These cases show how the power of algorithms can influence the lives of a very large number of people and restrict their rights through pervasive digital control.

Even if \pageinfoStart{12}we exclude the significant cases mentioned above, many small examples show how the impact of algorithms on the daily lives of ordinary people is \hbox{noteworthy} and pervasive. Among the many cases happening every day to a \hbox{myriad} of individuals whose lives are influenced by an algorithm, there is one very simple one that may explain how algorithms are pervasive. Miles Taverner is a 44-year-old artist living in Burnham-on-Crouch, Essex. During his beachcombing he often brought home pieces of driftwood, old bottles, corroded fragments of steel, and so on. As his collection grew, he started to think how he could shape it in art, depicting a unique story of travel around the seas and oceans. During one of his walks in February 2020, Taverner found a bottle that contained a sheet of paper with a phone number and a message: ``If found please contact Chloe/Alfie, much love, Clacton-on-Sea, England.'' Taverner tried to dial the phone number with his smartphone, but he discovered that it was no longer in use.

When Miles \pageinfoStart{13}Taverner checked the Facebook News Feed on his phone, he noticed that among the suggestions of potential new friends Facebook had included a certain Alfie Hillier. This person was in fact the one who had thrown the bottle into the sea eight years earlier. Hillier had registered his Facebook account years ago with his old phone number, exactly the one that was written in the message in the bottle. This mysterious suggestion happened because the algorithms of Facebook include collecting data from people's phones. Since 2009, when Facebook debuted a new sorting order for the News Feed based on a post's popularity, its EdgeRank algorithm has ranked the posts each user sees in the order that they are likely to enjoy them. When ranking posts, the algorithm uses a set of factors such as the relationship and proximity of a user, the content of a post, how new/old is a post, and so on. The same occurs for new friend suggestions that are linked to the actions/or information a user does/performs or stores on their smartphone. In this case, the algorithm of this social platform found that Taverner had tried to call a phone number that was similar to the one stored in the record of another of its users (Alfie Hillier). It then suggested that Taverner should try to make friends with Hillier.

``It's amazing \pageinfoStart{14}really. He popped up and was from the correct location and I just thought `it must be him' so I sent him a message and we had a good chat,''\break Taverner explained to the <c|i|t|e|t|C|n|t|-|1|3|8|1|0>, who told his curious story. Eight years before, Alfie Hillier lived in Clacton-on-Sea and worked on the pier. Hillier explained that he wrote the message together with a colleague named Chloe while they worked together. ``I was in complete shock when Miles contacted me. I thought it was a wind-up,'' he said. ``I never thought it would be found. It brings back lots of memories for me about my time working on the pier.''

Solving the \pageinfoStart{15}mystery was ``really exciting'' for Taverner. However, we know that the algorithm solved the mystery. In fact, Facebook collected metadata from his phone about his behavior and actions, such as phone calls and text messages. Then, as mentioned before, the Facebook algorithm searched the metadata and found that Miles Taverner had tried to call the same phone number stored in the record of another user (Alfie Hillier). After comparing the two phone numbers and finding they were similar, it suggested that Taverner make friends with Hillier.

This is \pageinfoStart{16}just one simple case that shows how social media algorithms know a lot about us as they monitor our lives through the digital devices within which they run. Some users discovered that if you were using an iPhone or an Android phone with the Facebook app installed, the social network had probably been \hbox{logging} your phone calls and text messages' metadata ever since the app's inception. Metadata is data that provides information about other data. For example, profile data are a case of actual ``data about our personal data.'' Metadata of phone calls may include information about the phone number of the caller, the call time, and the duration of a call. Metadata can be harvested, stored, and then exposed to algorithms that can use them to search related information on databases or on the web. They can then run complex operations on the basis of the collected metadata. In short, metadata are food for algorithms that perform complex procedures starting from them. In particular, by storing our metadata, the Facebook algorithm is able to track who and when we called, how long we were on the call, or when we texted. This simple case shows that people could be wrong in believing that it was destiny that connected Miles Taverner to Alfie Hillier via a message in a bottle. Actually, it was an algorithm that processed their personal data and metadata to discover private facts and relationships.

Algorithms---sequences of \pageinfoStart{17}operations that are executed by automated machines---{\allowbreak}are the main engines in automating processes. In doing so, they automate decisions that have an impact on people's lives. Like the Facebook case discussed here, every time we interact with a mobile app, a website, or a search engine, several algorithms running on our mobile phones, web servers, and social media platforms collect our data. Then machine-learning algorithms, such as the ones used by Facebook, Netflix, or Amazon, use them to profile us, to discover our preferences, our lifestyle, our job, or our health and well-being. Possessing this rich information, algorithms are able to provide us with suggestions, advice, advertisements, and other recommendations. In this way, they affect our decisions, sometimes more than our relatives or friends. They may exert influence on millions of people (although most of them don't know it). They are programmable dispensers of power.

Alexis de \pageinfoStart{18}Tocqueville in his book \textit{Democracy in America} (\textit{De La D\'{e}mocratie en Am\'{e}rique}) [<c|i|t|e|a|l|t|C|n|t|-|1|7|3|3|7>], originally published in the 1830s, discussed how public opinion in democratic republics may transform despotism to something less tangible: ``the power of the majority ... is an invisible, almost intangible power that makes a mockery of tyranny in all its forms.'' In this context, cultural domination may influence civil society through the power of thought affecting millions of citizens. No threat or physical coercion is needed to exercise this intangible power, which is as strong as traditional material power. A century later, Antonio \hbox{Gramsci} developed a similar notion by introducing the \textit{hegemony} concept. He used this word to explain the cultural hegemony of the ruling classes as a strong intangible power that is exercised over a nation. Today, algorithms, software, and Big Data are exercising great influence on billions of people. The impact is both massive and soft. Thus, many people do not realize that they are being influenced or even \hbox{manipulated.} For these reasons, being aware of the power of algorithms is a must for exercising the personal and civil rights of a full citizenship.

Every form \pageinfoStart{19}of mediation of the relationship with reality that humans have created during their evolution represents the construction of a new power. That power regulates the relationships among individuals and influences collective behaviors. Many algorithms today are designed to mediate our relationships with other people and with the world. Without a doubt, software programs implementing these algorithms represent very powerful authoritative and controlling mechanisms, more powerful than many others that mankind has experienced. Indeed, this power lies in the hands of those who design, implement, and disseminate algorithms in the form of software. This ``fills'' the operating logic of the Internet, our computers, and our smartphones. These are digital devices that we use. However, at the same time, they use us. They can do it because of the logic underlying them, and because of the complex and countless procedural rules that guide them.

In the \pageinfoStart{20}summer of 1985, Italo Calvino wrote the \textit{Six Memos for the Next Millennium} [<c|i|t|e|a|l|t|C|n|t|-|1|9|5|5|2>], a series of lectures written for the Charles Eliot Norton Lectures at Harvard. In the first lecture dealing with the value of lightness (\textit{leggerezza}), with great acuteness Calvino underlined the importance of software. The importance of his being light thinking which demonstrates, like other concepts studied by science (DNA, neurons, quarks, neutrinos), that the world is based on very subtle and at the same time fundamental entities. In the Lightness memo, Calvino wrote: ``It is true that software cannot exercise its powers of lightness except through the weight of hardware. But it is software that gives the orders, acting on the outside world and on the machines, which exist only as functions of software and evolve so they can work out ever more complex programs. The second industrial revolution, unlike the first, does not present with crushing images such as rolling mills or molten steel, but with `bits' in a flow of information travelling along circuits in the form of electronic impulses. The iron machines are still there, but they obey the orders of weightless bits.'' Italo Calvino\vadjust{\pagebreak} had well understood the role of algorithms and of their implementation through software programs. He clearly identified their great capacity as a formidable mediating agent, as a leading entity. The power of the lightness of software is much more important than the heaviness of hardware of which the ``iron machines'' that ``obey the weightless bits'' are made.

Computer scientists \pageinfoStart{21}and software developers know very well the definition of an algorithm provided by Donald Knuth, who argued that it refers to a ``finite set of rules that gives a sequence of operations for solving a specific type of problem'' [<c|i|t|e|a|l|t|C|n|t|-|2|1|3|3|1>]. Unfortunately, non-experts who use algorithms constantly in their daily lives do not have a good understanding of the underlying principles, features, and behaviors of these automatic procedures. Given algorithms' pervasiveness and relevance, it is time to understand the importance of a mature and responsible use of these sequences of instructions---a conscious use of these very long flows of operations that programmers impart to computers for producing answers or solving problems. When the instruction sequences are complex and highly sophisticated, they generate machine intelligence applications and data-driven learning systems that implement intelligent processes similar to what the human mind can do. However, these intelligent methods can be performed by billions of computers, smartphones, and digital machines that populate our world. Therefore, they are often much faster and more efficient than the human brain.

Learning the \pageinfoStart{22}nature of algorithms and the ways in which they manage and process data is crucial for understanding the concepts and practices of computer science. Knowing the techniques and the digital systems that allow their implementation as software programs is essential. In fact, this is necessary for grasping the rationale of these applications that today play such an important role in the functioning of our society. Most of the algorithms used today are based on the analysis of large datasets (the already mentioned Big Data) that represent stored experience: a warehoused know-how from which algorithms can learn new knowledge, gain insights, and propose future trends. By processing large sets of data, machine learning algorithms conceive actions and predict behavior models in many application domains. They are able to exploit the information hidden in Big Data for guessing a next move and/or for supporting complex decisions and strategies.

Big Data \pageinfoStart{23}is a technology that was born and raised with the Internet and grew with the extensive use of its software tools and services that communicate and share information. However, today Big Data is more than a technology or a repository of information for analysis and prediction. Coupled with algorithms, Big Data has become a universal model of gathering aspects of people's lives and also a way of thinking. In particular, Big Data analysis finds ways to express how personal and social decisions are made, and who can make them. This is useful in enabling businesses and research collaborations alike to make informed decisions. New discoveries are achieved and more accurate studies can be performed due to the increasingly widespread availability of large amounts of data. Scientific fields and business sectors that fail to make use of the huge amounts of digital data available today risk losing out on the significant opportunities that Big Data processing can offer.

However, a \pageinfoStart{24}new form of social, economic, and political control is built around Big Data. Collecting and processing the digital bits of our daily lives allows us to study many trends and behaviors. In this scenario, data analysis algorithms represent sophisticated methods to glean preferences and discover the behaviors of individual people or society in general. Algorithms and data processing machinery are the apparatuses that may implement the ongoing evolution of our world into a new space driven by data and by the people that own data. Business areas such as finance, retail, manufacturing, and healthcare have changed many of their processes and products through the use of Big Data, leading to better informed decision-making and improvements in corporate competitiveness, while many, significant scientific applications are now based on complex data analysis and machine learning algorithms (this occurs, e.g., in genomics, physics, linguistics, social sciences, and medicine).

Computers and \pageinfoStart{25}software programs were originally developed to help scientists solve mathematical problems and to speed up scientific discoveries. Now they are opening new scientific disciplines and are transforming several traditional fields of science and engineering. Computational linguistics, synthetic biology, data science, bioinformatics, and computational social science are just a few of the new scientific fields originated by the use of algorithms for modeling and simulating natural phenomena. While algorithms are transforming science, they are also driving financial trading decisions on Wall Street. In fact, software programs execute buy and sell orders on the basis of multifaceted algorithms, with no humans involved in the process. This means that portfolio managers are not making the trading decisions---computer scientists and programmers mostly drive the stock market.

Although algorithmic \pageinfoStart{26}practices are shaping our world more than ever before, the notion of algorithms and their real contribution in influencing our society are not so well known and widespread. Most people don't know the ways in which algorithms are implementing the machine intelligence that they use every day. The aim of this book is to introduce and analyze the basic concepts of algorithms and discuss the role of algorithms in ruling and shaping the world in which we live. The main goal is to help readers clearly understand the power and impact of this pervasive use of algorithms on human lives. Topics such as Big Data analysis, machine intelligence, job transformations, human value alignment, data science, automatic learning methods, social control, and the political use of algorithms are illustrated and discussed. In doing this, the book uses a descriptive writing style while making careful and appropriate use of technical concepts and issues.

The book \pageinfoStart{27}is intended primarily for non-specialist readers, and it does not assume a technical background in computer science. More specifically, the target audience of the book is the educated public interested in understanding what algorithms and machine intelligence are and how they are impacting our lives. Also, professionals dealing with digital technologies and young people attending information communication technology (ICT) classes or studying the societal impact of computer science will find the book beneficial. The book can also be used as a supporting text in undergraduate/{\allowbreak}graduate courses on digital transformation, introduction to computing, Big Data analysis, social media mining, computer ethics, and digital rights.

The first \pageinfoStart{28}four chapters set the ground by introducing and clarifying the basic concepts, evolution, and operational functions of an algorithm. Software systems, data management, and artificial intelligence concepts are presented and discussed. Chapter~\ExternalLink{chap:1}{1} introduces the concept of an algorithm, its principles and features. It then discusses the history and evolution of algorithms from the ancient definition up to modern times. Chapter~\ExternalLink{chap:2}{2} discusses how abstract algorithms can be transformed into software programs and how computers execute them. Algorithms and software features are compared while also explaining how these two concepts created new ways to address and solve problems. Chapter~\ExternalLink{chap:3}{3} examines the relationships and synergy between algorithms and data. In particular, Big Data concepts are also discussed through practical use cases in significant application domains such as digital health, smart cities, and finance. Chapter~\ExternalLink{chap:4}{4} discusses how machines may become ``intelligent'' by running learning algorithms on large datasets. The concepts of data-driven induction, artificial intelligence, machine learning algorithms, and data science are described.

Chapters~\ExternalLink{chap:5}{5}, \ExternalLink{chap:6}{6}, and \pageinfoStart{29}\ExternalLink{chap:7}{7} focus on the impact of algorithms on life and work and also discuss how values and biases can influence the results of algorithms. Chapter~\ExternalLink{chap:5}{5} discusses how algorithms are pervasive in our lives and how much they influence our way of living. Chapter~\ExternalLink{chap:6}{6} examines the relationships between algorithms and human work. It discusses how jobs are changing and how computers and their software are influencing the labor sphere and our working activities. Chapter~\ExternalLink{chap:7}{7} addresses the main issues of objectivity and ethics principles in software programming. This chapter also discusses the human value alignment of artificial intelligence algorithms that frequently run with increasing autonomy and proficiency in many real-world domains.

Chapters~\ExternalLink{chap:8}{8}, \ExternalLink{chap:9}{9}, and \pageinfoStart{30}\ExternalLink{chap:10}{10} emphasize the social and political aspects related to algorithms, data, and digital machines. In particular, Chapter~\ExternalLink{chap:8}{8} analyzes the role of algorithms in influencing political decisions and the functioning of democracy. Chapter~\ExternalLink{chap:9}{9} tackles the question of how big companies, governments, and autocratic regimes are using algorithms and machine intelligence methods to influence people, laws, and markets. Often, decisions made by algorithms are neither clear nor visible to users. At the same time, some computing models are obscure in their procedures. Chapter~\ExternalLink{chap:10}{10} discusses issues and methods related to algorithm transparency, accountability, and openness.

Finally, despite \pageinfoStart{31}their power, algorithms are not able to compute everything and solve every problem. Chapter~\ExternalLink{chap:11}{11} provides a discussion of the limits of computability. The chapter introduces the features of the non-computable and discusses some problems for which there is no algorithm able to solve them.

Digital technologies \pageinfoStart{32}are demonstrating their enormous power in all sectors of society. They allow us to be in contact always and everywhere. They help us at work, allow online teaching in schools and universities, and bring us digital content at all times and anywhere. Algorithms are very helpful in emergency management, in hospitals, in transport scheduling, and in finance and banks. Information \hbox{systems} and their algorithms play a key role worldwide. At the same time, they are permitting us to practice new relationships with everyday life, which in fact create new forms of citizenship. In short, they are transforming our world and human relations. At the same time, big companies such as Google, Apple, Facebook, and Amazon are among the most valuable companies globally. This is mainly because of the algorithms that they own. They sell services and products according to new models and strategies based on unique and complex algorithms. These strategies allow them to be very competitive and drive out of business conventional companies that have been unable to maximize the benefits of digital automation and software technologies.

This is \pageinfoStart{33}an unprecedented situation that is changing the present and future of all citizens. However, each new tool is not fully neutral---more so if it is very complex and not only generates benefits but requires greater awareness and suitable rules and laws. These elements are needed to avoid situations where an algorithm may favor someone to the detriment of others. As mentioned before, digital technologies have allowed the creation of large technological empires. They now have a larger influence and a more important role than that of some governments. As a consequence, decisions on the utilization of their algorithms and the data they collect, affecting billions of people, cannot be left only to their CEOs or CTOs. For many years now, big tech companies\vadjust{\pagebreak} such as Google and Facebook, although\break providing information and services to billions of users, have collected user data and profiled those users; likewise with Amazon and Netflix, whose learning algorithms trace user behaviors and use the data to provide personal advertising or to improve market targeting. Customers must be aware of how algorithms use their data and how they manipulate critical information targeting millions of people. They must be conscious of how automatic data analysis influences their behavior, purchases, and opinions. They must also be conscious that applying computing technology to every aspect of our lives does not solve all human problems. As argued by Meredith Broussard, although digital technology is powerful and flexible, there are fundamental limits to what we can do with its tools [<c|i|t|e|a|l|t|C|n|t|-|3|3|8|9|4>]. Understanding the limits of what we can do with computers is useful since knowing the limits of digital computing systems allows people to make better use\break of~them.

The social \pageinfoStart{34}psychologist Shoshana <c|i|t|e|t|C|n|t|-|3|4|1|3|4> has coined the term ``surveillance capitalism'' to describe this scenario. This new economic model works by providing free services to billions of people, consequently providing the providers of these services the leeway to intensely monitor the sentiment of customers. Unfortunately, consumers often ignore the algorithmic procedures they use and habitually do not provide their explicit consent to give up their data. The surveillance capitalism model has been extended to politics in recent years. There are \hbox{several} cases that show how governments, political parties, and sometimes even individual politicians have embraced the model of citizen surveillance, and how these entities have graduated from surveillance to manipulation. The cases mentioned previously of Russian bots, the Trump--Cambridge Analytica scandal, and the agreement between the Chinese government and Alibaba for exploiting \hbox{customer} data in the Social Credit System are very significant examples. They show how, through the use of algorithms, politics is taking over the Internet and social media communication spaces to appropriate the model of surveillance capitalism for political purposes. Some parties or governments are practicing control and manipulation through data collection and analysis algorithms. They inundate the network of emails, posts, and tweets to steer opinions. They generate fake news, offenses, prejudices, threats, and other ``digital atoms'' that led us into the ``psychopolitics'' phase, well described by the philosopher Byung-Chul <c|i|t|e|t|C|n|t|-|3|5|7|1|6>.

To live \pageinfoStart{35}up to the full potential of the creative and research capacity that humans have developed, humans must also increase the knowledge of individuals and their skills in the ethical and social use of scientific discoveries. Knowledge, critical thinking, social control, and realism are needed. The great ability to innovate that human beings have nurtured has irrevocably changed the world and humanity itself. This scenario requires pragmatic skills for handling the potential negative impacts of these innovations on humanity. Algorithms and digital technologies will surely play a primary role in future scenarios. Therefore, extending knowledge and awareness about them is essential for understanding our future and acting in it as protagonists for the benefit of society.

<a|d|d|s|e|c|C|n|t|-|3|6|5|2|9>

I wish \pageinfoStart{36}to thank my colleagues in the SCALAB laboratory of DIMES at \hbox{University of} Calabria for many useful discussions on the book topics over the years. My thanks are also due to Sean Pidgeon, Bernadette Shade, Brent Beckley, Karen Grace,\break Barbara Ryan, Julia Stevenson and the anonymous reviewers for their useful\break support and comments on early drafts of the book and during all the editorial process. I would like to thank Enza, Marianna and Francesco for their tolerance during the writing of this book. Writing a book is a complex and time consuming task, without the help of many persons it cannot be accomplished successfully.

<t|h|e|b|i|b|l|i|o|g|r|a|p|h|y|C|n|t|-|3|7|2|0|6>

<L|a|t|e|x|C|o|m|m|e|n|t|C|n|t|-|3|8|4|4|3>

